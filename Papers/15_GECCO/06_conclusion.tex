\section{Conclusion}
% discrete problems seem harder for the GRN... we need to say that in the conclusion

In this paper, we employ gene regulatory networks to serve as neuromodulators of learning parameters for a reinforcement learning algorithm on four benchmark problems. We have evolved GRN specifically on each task, which generally produced better or equivalent results to the standard fixed-parameter SARSA algorithm. In all cases, the worst case scenario is always better handled by genetically-regulated neuromodulation due to its ability to perform on-the-fly parameter adaptations which reduces complications of learning with fixed parameters, such as incorrect credit assignment in sequences of actions.

Additionally, the GRN evolved on the Maze problem as well as a generic GRN have been tested on the problems and offers encouraging results for multi-task learning in multiple problem domains. Once again, the worst-case scenario is consistently handled better by genetically-regulated neuromodulation with generic GRNs than with fixed-parameter algorithms. However, generic GRNs are not as good as GRN specifically optimized for each task. Other inputs might be useful to the GRN for estimating the quality of the current neuromodulation and the quality of the agent behavior. The pursuit a generic GRN capable of generalizing the regulation of learning parameters across multiple tasks and problem domains would not only eliminate the need to search for additional GRNs, but also the parameter sampling phase commonly required for reinforcement learning applications.

%\section{Future Work}
As was noted in \cite{Schweighofer2003}, neuromodulation is neither problem nor RL-algorithm specific, thus future work may investigate the application of genetically-regulated neuromodulation to alternative RL algorithms. We have shown that a simple feedback controller can optimize community features of agent-based swarm simulations, the use of GRN-based neuromodulation is likely to further optimize these environmental controllers \cite{Gold2014}. \newline

% Removed that last bit from future work because I think we have it here ; )

\textbf{Acknowledgements:} Kyle Harrington is supported by NIH grant 2T32HL007893-16A1.
