\section{Gene Regulatory Network}
%%%% FROM IJCNN %%%%

Artificial gene regulatory networks (GRNs) are a class of biologically-inspired
algorithms. In living systems, gene regulatory networks are used within the cell
to control DNA transcription and, correspondingly, the phenotypic gene
expression. Although the inner workings of the cell are governed by a large collection
of complex machines, simplified models of cells as entities with protein sensors and actuators
both exhibit complex behavior and offer insights into natural systems \cite{Davidson2006}.
These protein sensors represent receptor molecules localized to the cellular membrane
which transduce external activity into excitatory and/or inhibitory regulatory signals.
Cells use external signals collected from protein sensors localized
on the membrane to activate or inhibit the transcription of the genes.

Our computational model uses an optimized network of abstract proteins. The inputs of the
agent are translated to protein concentrations that feed the GRN. Output
proteins regulate the reinforcement learning parameters previously described.
This kind of controller has been used in many developmental models of the
literature \cite{Joachimczak08, Doursat09, CussatBlanc2012a} and to control
virtual and real robots \cite{ziegler2001evolving, Nicolau10, Joachimczak10, CussatBlanc2012b}. 

In our model, a
gene regulatory network is defined as a set of proteins. Each protein has the
following properties:
\begin{itemize}
\item The \emph{concentration} is the quantity of each protein avalaible in the network. This concentration influences the regulation of other proteins: the higher the concentration, the higher the enhancing or the inhibiting impact on other proteins.

\item The protein \emph{tag} coded as an integer between 0 and $p$. The
	upper value $p$ of the domain can be changed in order to control the
	precision of the GRN. In Banzhaf's work, $p$ is equivalent to the size
of a site, 16 bits. 
% We have reduced the precision to 16 because the
% precision of the output values are not a particular requirement in that
% particular application.

\item The \emph{enhancer tag} coded as an integer between 0 and $p$. The
	enhancer tag is used to calculate the enhancing matching factor
	between two proteins.

\item The \emph{inhibitor tag} coded as an integer between 0 and $p$. The
	inhibitor tag is used to calculate the inhibiting matching factor
	between two proteins.

\item The \emph{type} determines if the protein is an \emph{input} protein, the
	concentration of which is given by the environment of the GRN and which
	regulates other proteins but is not regulated, an \emph{output} protein,
	the concentration of which is used as output of the network and which is
	regulated but does not regulate other proteins, or a \emph{regulatory}
	protein, an internal protein that regulates and is regulated by other
	proteins.

\end{itemize}

The dynamics of the GRN is calculated as follow. First, the affinity of a
protein $a$ with another protein $b$ is given by the enhancing factor
$u^{+}_{ab}$ and the inhibiting $u^{-}_{ab}$:
\begin{equation}
u^{+}_{ab}=p-|enh_a-id_b|~~;~~u^{-}_{ab}=p-|inh_a-id_b|
\end{equation}
where $id_x$ is the tag, $enh_x$ is the enhancer tag and $inh_x$
is the inhibiting tag of protein $x$.

The GRN's dynamics are calculated by comparing the proteins two by two using the
enhancing and the inhibiting matching factors. For each protein in the network,
the global enhancing value is given by the following equation:
\begin{equation}
g_i=\frac{1}{N}\sum_j^N{c_je^{\beta (u^{+}_{ij}-u_{max}^{+})}}~~;~~h_i=\frac{1}{N}\sum_j^N{c_je^{\beta (u^{-}_{ij}-u_{max}^{-})}}
\end{equation}
where $g_i$ (or $h_i$) is the enhancing (or inhibiting) value for a
protein $i$, $N$ is the number of proteins in the network, $c_j$ is the
concentration of protein $j$ and $u_{max}^{+}$ (or $u_{max}^{-}$) is the
maximum enhancing (or inhibiting) matching factor observed. $\beta$ is a
control parameter described hereafter.

The final modification of protein $i$ concentration is given by the following
differential equation:
\begin{equation}
\frac{dc_i}{dt}=\frac{\delta(g_i-h_i)}{\Phi}
\end{equation}
where $\Phi$ is a function that keeps the sum of all protein concentrations
equal to 1.

$\beta$ and $\delta$ are two constants that set up the speed of reaction of the
regulatory network. In other words, they modify the dynamics of the network. $\beta$
affects the importance of the matching factor and $\delta$ affects the level of
production of the protein in the differential equation. The lower both values,
the smoother the regulation. Similarly, the higher the values, the more sudden
the regulation. 
